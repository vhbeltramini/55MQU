\documentclass[12pt]{article}

\usepackage{sbc-template}
\usepackage{graphicx,url}
\usepackage[utf8]{inputenc}
\usepackage[brazil]{babel}
\usepackage[latin1]{inputenc}  

\sloppy

\title{Múltiplo Caixeiro Viajante}

\author{Victor Hugo Grabowski Beltramini\inst{1},\\ André Ellias Zanella\inst{2}}
\address{Universidade do Estado de Santa Catarina (UDESC) \\ Ibirama, SC, Brasil
  \email{vhbeltramini@gmail.com, andreellias19@gmail.com}
}

\begin{document}

\maketitle

\begin{abstract}
  This article addresses a complex optimization problem, which consists of finding the best route for a network of brand representatives to travel through cities, starting from the same city and ending in the same city so that all cities be covered in the shortest possible distance. This article aims to solves this optimization problem by using a set of metaheuristics that contemplate constructive and modification strategies. Finally, a study is carried out experimental based on the results obtained.
\end{abstract}

\begin{resumo}
  Este artigo contempla um problema de otimização complexo, que consiste em encontrar a melhor rota para um grupo de N caixeiros percorrerem C cidades, partindo da mesma cidade e terminando na mesma cidade de forma que todas as cidades sejam percorridas na menor distância possível, este problema recebe o nome de caixeiros viajantes múltiplos. O presente artigo tem como objetivo resolver esse problema de otimização, por meio da utilização de um conjunto de metaheurísticas que contemplam estratégias construtivas e de modificação. Por fim é realizado um estudo experimental com base nos resultados obtidos.
\end{resumo}

\section{Introdução}

O problema do caixeiro viajante múltiplo (mTSP) é uma verção mais complexa do problema clássico de otimização que é o caixieiro viajante, este problema se basei em definir uma rota que tenha o menor costo para que um caixeiro que percorra um numero x de cidades, e na variação abordada nesse trabalho podemos ter n caixeiros ao invés de um com no problema inicial. Esse problema pode ser aplicado em diversos casos práticos, como planejamento de caminho tridimensional, planejamento de caminhos para robôs e drones ou serviço de transporte de contêineres.

O objetivo principal do trabalho é criar metaheurísticas que possam de forma efetiva diminuir o custo da função objetiva até um nível aceitável de satisfação. Para isso foram criadas 2 heurísticas, 2 buscas locasis e 2 GRASP. Ao final é feita uma análise de todos os dados obtidos para se obter uma melhor compreensão da proporção de melhora obtida.

\section{Problema} \label{sec:firstpage}

Este problema contempla o dificuldade de um grupo de indivíduos precisem passar por um grupo de cidades, de modo que todas as cidades sejam percorridas por no máximo um indivíduo e que todos iniciem e terminem na mesma cidade. Com isso, no final desejá-se obter a solução em que as soma das distâncias da rota de cada indivíduo seja a menor possível.

Para otimizar esse problema foi utilizada uma representação em grafo, onde cada vértice é uma cidade e as arestar representam a distância entre as cidades. Assim ao somar as distâncias do caminho de cada representante obteremos a função objetivo. Com isso obtemos os dados onde i é o númeor de cidades, t é o número de caixeiros e Dij é a matriz de distâncias entre cada cidade. A seguir temos o modelo matemático do problema:

\begin{figure}[ht]
\centering
\includegraphics[width=0.5\textwidth]{modelo.png}
\caption{Modelo matemático}
\label{fig:heuristica}
\end{figure}

\section{Algoritmos}

Inicialmente foram feitas 2 heurísticas construtivas, onde a primeir utiliza o critério de seleção como a menor soma encontrada para a distância entre a posição atual de cada caixeiro e a cidade c. Após achar a melhor soma (cidade) é então escolhido o caixeiro com a menor distância para essa cidade e que incluirá essa cidade na rota.

Para a segunda heurística foi utilizado o critério de seleção como a menor distância encontrada entre os caixeiros em suas posições atuais, incluindo assim essa cidade para esse caixeiro. Além disso para cada heurística foram feitas 2 implementações, sendo o primeira a gulosa, onde só é pego a menos distância encontrada e a segunda uma semi-gulosa do tipo alfa-gulosa, onde é pego um valor aleatório entre os 50\% melhores valores obtidos segundo cada respectivo critério.

Também foram feitas 2 buscas locais, as duas buscas locais tiveram como critério de aceitação somente soluções que melhorassem a solução incial, para que essa solução melhorada pudesse ser atingide foram implementadas duas maneiras de perturbações diferentes, uma delas sendo um shift entre rotas e a segunda aplicando um shifi intra-rotas, após cada uma dessas melhoras era definida está nova solução como sendo melhor, e portanto assim o algoritmo seguia com essa nova solução afim de enconcontrar uma solução ainda melhor.

Por fim foram desenvolvido e aplicados 4 algoritmos GRASP(Greedy Randomized Adaptative Search Procedure), combinando as duas heurísticas construtivas na sua implementação semi-gulosa com as duas buscas locais. A Heurística foi utilizada para a solução inicial e depois aplicado a busca local por uma quantidade x de vezes de forma que se obtenha uma melhor solução do que usando cada um de forma individual.

No implementações relacionadas ao GRASP o sistema se comportava da seguinte maneira, a partir de uma solução inicial advinda de uma heurísticas construtiva semi-gulosa era aplicada uma busca local afim de melhorar essa solução inicial e então era definida está como a melhor solução até o momento, após esta etapa inicial o sistema iterava x vezes realizando uma nova heurísticas construtiva semi-gulosa e novamente uma nova busca local a fim de se tentar encontrar uma solução melhor, caso não encontratava o sistema continuava com a melhor solução até o momento, e ao final das x iterações era retornada a melhor solução encontrada dentre as iterações.

\section{Resultados}

Segue abaixo a tabela com os resultados obtidos, onde na figura 2 contém as duas heurísticas desenvolvids com seus respectivos resultados otimizados para cada instância. Na figura 3 é constado as 2 buscas locais e os 4 GRASPs implementados, cada um com o seu resltado para as instâncias utilizadas. Ao final temos uma coluna com o GRASP deselvoido por \cite{alvesa:84} (figura 3), que foi usado como referência para a comparação de resultados obtidos. Para cada metaheurística foi pego o resultado com a melhor otimização obtida (S*) e a média entre 3 resultados obtidos (S). As instâncias utilizadas durante a implementação e teste fora obtidas de \cite{hehao:22}, onde as instâncias variam de 51 á 200 cidade e de 3 ou 5 caixieiros.

\begin{figure}[ht]
\centering
\includegraphics[width=0.5\textwidth]{heuristica1.png}
\caption{Heurísticas}
\label{fig:heuristica}
\end{figure}

\begin{figure}[ht]
\centering
\includegraphics[width=\textwidth]{heuristica2.png}
\caption{Buscas locais e GRASPs}
\label{fig:heuristica}
\end{figure}

Como visto nos resultados, ainda há uma difença considerável entre o melhor resultado encontrado a partir da nossa implementação e a melhor solução já encontrada, porém tal diferença pode ser diminuida considerávelmente a partir uma implementação mais otimizada da função responsável pelo cálculo da função objetiva que avalia a cada iteração se houve ou não uma melhoria na solução.

Como solução usada atualmente não está fazendo essa avaliação da função objetiva de maneira mais otimizada possível a mesma acaba custando um tempo considerável considerando o tempo total de execução para a definição de cada solução, isso acabou fazendo com que fosse diminuida o numero de iterações até se encontrar a melhor solução afim de que o processo não levasse muito tempo para entregrar uma solução melhor.

\section{Conclusão}

Com base nos resultados obtidos ao executar cada meta-heurística podemos perceber que a otimização de fato ocorreu, todas as instâncias apresentaram uma melhoria considerável. Único porém no entando é que quando olhando exclusivamente as buscas locais e as heurísticas construtivas ainda possívelmente há melhorias que podem ser feitas para que se chege mais proximas as melhores soluções já encontradas até o momento.

Portanto pode se concluir que as implementações atingiram sua proposta inicial que era de encontrar uma solução para o problema e ainda posteriomente foram capazes de melhorar solução uma inicial semi-gulosa que acaba trazendo um pouco de aletoriedade para a solução mas ainda permitindo encontrar uma solução boa a fim de otimizar ao máximo possível a rota dos caixeiros para que fosse percorrida com a menor distancia possível.

\bibliographystyle{sbc}
\bibliography{sbc-template}

\end{document}
